\section{Results}

\subsection{Data}

FASTA format tools were tested on the Vicugna\_pacos-2.0.1 whole genome shotgun sequence scaffolds because the 2.17 Gb \textit{Vicugna pacos} genome is large (> 1 Gb) and has many scaffolds (276727) \cite{Lindblad_Toh_2011}. The large genome size and high number of individual sequences should approximate a typical large FASTA file. The FASTA file was downloaded from the NCBI FTP as NW\_005882702.1 \textit{Vicugna pacos} isolate Carlotta (AHFN-0088) Vicugna\_pacos-2.0.1 assembly scaffolds.
An additional unwrapped sequence was added to the end of the file. This sequence was also missing a newline. Each FASTA record in the file also had spaces within the text of the headers.

Additional simulated FASTA record (backslashes are used to indicate a new line that is for display in the article rather than the new lines being included in the actual unwrapped FASTA record):
\begin{verbatim}
>NW_000000000.0 Vicugna pacos isolate Carlotta (AHFN-0088) FAKE genomic scaffold, Vicugna_p\
acos-2.0.1 Scaffold-, whole genome shotgun sequence
ATACAACCATAAAGGTGCTATTCAGTCCATGGTTACAGGACATAACTACAACACACACCCACGTACACATGCGCATGCGCATGCACACACC\
CACGTACACGTACACGTACGCATACACACCCACGTACACGTACACGTACGCATACACACCCACGTACACGTACACGTACGCATACACACCC\
ACGTACACGTACACGTACGCATACACACCCACGTACACGTACACGTACGCATACACACCCACGTACGCACACACGTACACGTGTAGGCACG\
CATTTAGCAAGTATTTAGCTTGCTTAAACAAACCCCCCCTACCCCCCACGAGCCCCACCTTATATACCAGACAGTCTTGCCAAACCCCAAA\
AACAAGACATAGCGCATAAGCTATAGAACCCGGACAAACCTTTGCCCACAAACCCAACTTCTTAAATAATCACATGGCCAAATCGTACCAA\
TGTGTTACTCTAGTATATTAAAAATATACAGACAGCTATCTCCCTAGATCCGCCAAAATTTTTAAAACAGAATTCAACAACCTTTTTAATG\
GCACCCCCCCCCCCCATAAATGACC
\end{verbatim}

\subsection{Reformatting tests}
No tool was found with all of Fasta-O-Matic's functions. Therefore sequence line wrapping was compared between Fasta-O-Matic and two other common reformatting tools, seqtk and seqret. Fasta-O-Matic was run with the \verb|--qc_steps| flag set to either \verb|wrap new_line header_whitespace| (all), \verb|wrap| (W) \verb|new_line| (NL) or \verb|header_whitespace| (HW). Seqtk was run with the arguments \verb|seq -l 60|. Seqret was run using only the \verb|-sequence| and \verb|-outseq| arguments. Code used in tests or to produce figure can be found on \href{https://github.com/kstatebioinfo/Fasta-O-Matic-a-tool-to-sanity-check-and-if-needed-reformat-FASTA-files/tree/master/figures}{github}. Run time and max memory was reported for each tool. Tests were run on a Xeon Phi server with 48x12-core Intel Xeon CPUs, 256GB of RAM, Linux CentOS 7 and Python2.7.

\subsection{Comparison between results}

All tools could reformat the improperly wrapped FASTA file. Fasta-O-Matic had the lowest maximum memory requirements (Figure 1, Table 1). This may be useful if working on a large genome on a local machine or cluster headnode where memory usage is restricted. Fasta-O-Matic took several minutes rather than seconds (seqtk and seqret took <13 s) (Figure 2, Table 1). 

Fully re-formatted simulated FASTA record (backslashes are used to indicate a new line that is for display in the article rather than the new lines being included in the actual FASTA record):
\begin{verbatim}
>NW_000000000.0_Vicugna_pacos_isolate_Carlotta_(AHFN-0088)_FAKE_genomic_scaffold,_\
Vicugna_pacos-2.0.1_Scaffold-,_whole_genome_shotgun_sequence
ATACAACCATAAAGGTGCTATTCAGTCCATGGTTACAGGACATAACTACAACACACACCC
ACGTACACATGCGCATGCGCATGCACACACCCACGTACACGTACACGTACGCATACACAC
CCACGTACACGTACACGTACGCATACACACCCACGTACACGTACACGTACGCATACACAC
CCACGTACACGTACACGTACGCATACACACCCACGTACACGTACACGTACGCATACACAC
CCACGTACGCACACACGTACACGTGTAGGCACGCATTTAGCAAGTATTTAGCTTGCTTAA
ACAAACCCCCCCTACCCCCCACGAGCCCCACCTTATATACCAGACAGTCTTGCCAAACCC
CAAAAACAAGACATAGCGCATAAGCTATAGAACCCGGACAAACCTTTGCCCACAAACCCA
ACTTCTTAAATAATCACATGGCCAAATCGTACCAATGTGTTACTCTAGTATATTAAAAAT
ATACAGACAGCTATCTCCCTAGATCCGCCAAAATTTTTAAAACAGAATTCAACAACCTTT
TTAATGGCACCCCCCCCCCCCATAAATGACC
\end{verbatim}
  
  
  
  
  
  
  
  
  