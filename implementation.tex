\section{Implementation}

In order to fit seamlessly into an analysis workflow a FASTA format tool should should be able to detect which format issues are actually present in the FASTA file. It should only produce a reformatted file if the current file violates the user defined format. 


\subsection{Portability}

Where possible Fasta-O-Matic was designed to be easy to distribute and use. Fasta-O-Matic is distributed on GitHub under a the MIT license to allow for easy access to or customization of the code. The tool was also built and tested on both Python2.7 and and Python3.3 to minimize incompatibility with existing linux environments. The script generates complete help menus when called from the command line with the \verb|--help| command and from within python with \verb|help(fasta_o_matic)|. Additionally, Fasta-O-Matic includes a sample FASTA file with missing newlines, inconsistent wrapping and spaces in headers and a tutorial which describes how to reformat the sample. These features ensure that Fasta-O-Matic is easy to incorporate into existing workflows.

\subsection{Automate where appropriate} 

The script was designed to efficiently execute the most likely solution given the presence or absence of format issues. Fasta-O-Matic returns the filename of the FASTA file that conforms to the user defined format. If the original file already conforms then Fasta-O-Matic returns the original filename rather than outputting a redundant FASTA file under a new name.

Fasta-O-Matic will exit and report an error if the FASTA file cannot be read, the default or defined output directory cannot be written to or the input FASTA file does not begin with a \verb|>|. The last error is considered the only fatal FASTA format error. \verb|ADD TEST FOR IUPAC BASES AND AMINO ACIDS AND DASHES IN SEQUENCE (NON-HEADER) LINES AS THE SECOND MAJOR FATAL ERROR|.

Inconsistent or unwrapped sequence lines, spaces in headers and missing or non-standard new lines are considered non-fatal errors. If they are detected the decision is made to reformat as requested, report the issue to the analyst and continue the workflow.

The script also automatically adjusts to run the minimal number of steps sufficient to fix and report format issues. If it is included in the set of QC steps then wrapping is the first format issue tested because while repairing FASTA wrapping both headers and new lines can be corrected. New lines are given priority after wrapping because while repairing new lines it is also trivial to repair headers. Finally, headers are evaluated for format issues. If an early test returns a format issue and launches a reformatting that automatically repairs any remaining format issues Fasta-O-Matic still tests for any additional format errors in the original file. The analyst should be made aware of any unexpected format issues in case they indicate an unexpected issue with the data.
  
  
  
  
  
  