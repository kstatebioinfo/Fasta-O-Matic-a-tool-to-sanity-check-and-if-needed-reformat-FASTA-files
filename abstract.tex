As the shear volume of bioinformatic sequence data increases the only way to take advantage of this content is to more completely automate robust analysis workflows. Analysis bottlenecks are often mundane and overlooked processing steps. For example, when data generation outpaces data transfer speed processing literally can move no faster than the speed of shipping a hard drive. Likewise, idiosyncrasies in reading and/or writing bioinformatics file formats can halt or impair analysis workflows by interfering with the transfer of data from one informatics tools to another. If file format incompatibility requires an analyst to intervene manually and assess the severity of the formatting issue, reformat the file and restart automated analysis then processing a job to completion will be no faster than the time it takes for the analyst to intervene. This need for automation must also be balanced by the need for humans to be able to confirm that any formatting error was actually minor rather than indicative of a corrupt data file. 

To that end, Fasta-O-Matic runs customizable quality checks of the FASTA file format. File format errors are categorized as fatal or non-fatal by Fasta-O-Matic.  If all issues are classed as non-fatal, the tool will repair improperly formatted files, return the file name of the reformatted file and report any issues detected to the user based on a user-defined list of quality checks. If a fatal formatting issue is found Fasta-O-Matic will exit and report the error without creating a reformatted file. If no error is found no new file is created.

Fasta-O-Matic was designed to prepare adapter FASTA files for Trimmomatic but can be used as a general pre-processing tool in bioinformatics workflows (e.g. to automatically wrap FASTA files so that they can be read by BioPerl). The program, Fasta-O-Matic, was also developed as a sanity check for bioinformatic core facilities that tend to repeat common analysis steps on FASTA files received from disparate sources. Fasta-O-Matic could be set with format requirements specific to downstream tools and could be added as a first step in larger analysis workflows.

The Fasta-O-Matic script also has optional color coded and quiet/verbose (displays only warnings and errors or also displays informative messages) logging settings. These were designed to draw the attention of the bioinformatics analyst to relevant warnings or errors faster even if they have grown accustomed to seeing Fasta-O-Matic output frequently.

Fasta-O-Matic is available free of charge to academic and non-profit institutions at \url{https://github.com/i5K-KINBRE-script-share/read-cleaning-format-conversion/tree/master/KSU\_bioinfo\_lab/fasta-o-matic}.
  
  