Fasta-O-Matic runs quality checking of the FASTA file format. File format errors are catagorized as fatal or non-fatal by Fasta-O-Matic. The tool will repair improperly formatted files, return the file name of the reformatted file and report the issue to the user based on a user-defined list of quality checks if the issues are non-fatal. If a fatal formatting issue is found Fasta-O-Matic will exit and report the error without creating a reformatted file. If no error is found no new file is created.

Fasta-O-Matic was designed to prepare adapter FASTA files for Trimmomatic but can be used as a general pre-processing tool in a bioinformatics workflow (e.g. to automatically wrap FASTA files so that they can be read by BioPerl).
