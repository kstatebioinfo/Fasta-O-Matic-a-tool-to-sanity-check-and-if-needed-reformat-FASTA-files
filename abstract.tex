As the shear volume of bioinformatic sequence data increases it becomes increasingly true that the only way to take advantage of this content is to more completely automate robust analysis workflows. Analysis bottlenecks are becoming the  mundane and often overlooked steps. For example, as data generation out paces data transfer capabilities processing literally can move no faster than the speed at which one can physically deliver a hard drive. Likewise, idiosyncrasies in reading and/or writing bioinformatics file formats can halt or impair analysis workflows by interfering with the transfer of data from one processing tools to another.

Fasta-O-Matic runs customizable quality checks of the FASTA file format. File format errors are categorized as fatal or non-fatal by Fasta-O-Matic.  If all issues are non-fatal, the tool will repair improperly formatted files, return the file name of the reformatted file and report any issues to the user based on a user-defined list of quality checks. If a fatal formatting issue is found Fasta-O-Matic will exit and report the error without creating a reformatted file. If no error is found no new file is created.

Fasta-O-Matic is designed to prepare adapter FASTA files for Trimmomatic but can be used as a general pre-processing tool in bioinformatics workflows (e.g. to automatically wrap FASTA files so that they can be read by BioPerl). It may be useful to bioinformatic core facilities that tend to repeat common analysis steps on FASTA files received from disparate sources. Fasta-O-Matic could be set with format requirements specific to downstream tools and could be added as a first step in analysis workflows.

Fasta-O-Matic is available free of charge to academic and non-profit institutions at \url{https://github.com/i5K-KINBRE-script-share/read-cleaning-format-conversion/tree/master/KSU\_bioinfo\_lab/fasta-o-matic}.
  