
\section{Introduction}

Sequence data can be stored as text with each letter representing a nucleic acid (DNA and RNA) or amino acid (protein). The linear nature of these molecules makes it natural to represent them as strings, finite sequences of characters. Although it has been argued that a graph, a network of edges connected by vertices, is a more accurate way to store genomic sequences because graphs allow the inclusion of alternate alleles and alternate possible assemblies \cite{jaffe2012fastg} all of the most common methods for storing sequences (FASTA, FASTQ, SAM/BAM) use a linear strings.

Other decisions about how to represent sequence data can be more arbitrary. For example, any character that is not used as base or an amino acid can be used to indicate the beginning of a new sequence. Additionally text can be wrapped to limit the information content in any one line of a file. The advantage of wrapping text is that some programs can then be designed to work one line at time limiting the burden of each step (e.g. the program would never have to process an entire chromosome of sequence data in a single step). The disadvantage is that code must be slightly more complex to load an entire sequence record into the working memory.

\subsection{FASTA file format specifications versus recommendations}

FASTA file format requirements are very minimal \cite{FASTAformat}. Each sequence is preceded by a header/description line that begins with a \verb|>| symbol. Sequence lines can include any standard IUB/IUPAC single character symbols for nucleic acids or amino acids or the ambiguous codes that indicate possible residues or bases \cite{comm1970abbreviations}. They can also include \verb|-| to indicate alignment gaps and \verb|*| to indicate stop codons. 

NCBI recommends wrapping FASTA file sequences lines \cite{FASTAformat}. It is also common practice to use the first `word' in a header (i.e. any character string to the left of the first space in the header) as the unique sequence id. Although these features are common they are not required leading to format compatibility issues with tools that treat these conventions as required.

\subsection{Customizing FASTA files to ensure that information is properly interpreted by downstream tools}

Regardless of whether a FASTA file is technically improperly formatted or it's format merely violates a popular convention it is critical to quality analysis workflows that data is converted into a format that will be correctly interpreted by downstream tools. Formatting issues can fall into multiple categories including actual format errors, formats that are not technically wrong but are non-standard and formats that throw errors because an existing tool has a bug (in which case we should modify the FASTA and proceed only if the tool will then correctly import the data and export the desired output). 

Some format errors indicate a major problem like an attempt to use the wrong data format (e.g. the first line is not a FASTA header because it does not begin with a \verb|>| character). These types of errors will be subsequently referred to as fatal. Alternately, some formatting issues occur commonly without indicating the FASTA file is corrupt (e.g. improperly wrapped/unwrapped sequence lines, missing final new line characters, unusual new line characters like \verb|\r|). These issues will be referred to as non-fatal. Fatal formatting issues should cause processing to stop. Non-fatal formatting issues should be automatically corrected according to the most common resolution for this type of error. While downstream processing continues the analyst can double check the automated decision to reformat non-fatal issues. This way workflow would not be slowed for trivial reformatting steps and the more rare problems (e.g. when a missing last new line was caused by incomplete file transfer) could still be caught.
  
\subsection{Existing tools}

Many bioinformatics tools already address FASTA format inconsistencies. However many tools either halt and exit with an error (e.g. BioPerl) or can produce can reformatted output FASTA but cannot determine if there is a formatting issue to begin with (e.g. EMBOSS Seqret). 

The BioPerl module \verb|DB::Fasta| will halt if a FASTA is inconsistently wrapped or if a line of sequence is too long (as in an unwrapped genome FASTA). This has the disadvantage of requiring intervention to wrap and restart analysis.

Code:
\begin{verbatim}
#!/usr/bin/perl
use Bio::Seq;
use Bio::SeqIO;
use Bio::DB::Fasta; #makes a searchable db from my fasta file
my $out_file_temp = '/home/bionano/test_db/all.fa';
my $seq_out = Bio::SeqIO->new('-file' => ">$out_file_temp",'-format' => 'fasta');#Create new 
#FASTA outfile object.
my $db = Bio::DB::Fasta->new("/home/bionano/test_db/miswrapped.fa");#Load FASTA file as DB
my $seq_obj = $db->get_Seq_by_id('seq'); # get FASTA records using headers 
#(where header = first 'word' so really header whitespace should also be removed for this file)
$seq_out->write_seq($seq_obj);
\end{verbatim}

Input: 
\begin{verbatim}
>seq 1
ACTGTGTGCAATCGCTGNNNNCTCTCATCGGATCTTGCAATCGCTNNNCTCTCATCGGATTGCAATCGCTNNNCTtcatcCGGAT
CGCTGNNNNCTGTGTGCAATCGCTGNNNNCTCCTGATCGCTGNNNNCTGTGTGCAATCGCTGNNNNCTCCTGCAATCGCTGNNNN
CTCCTGTTCGNATCGatcctctgtttatgcttatagctagctgatcgtagnnntcaacgt
CTAGAGCGCAGCTCTGGGGGATTACTACTCACTACATCATTAGATCAGATacgactcann
>seq 2
cttatagctagctgatAATCGCTGNNTCATCGGATCTTGCCTTGCAATCGtcatcCGtcC
CGCTGNNNNCTGTGTGCAnnnnnnnnnnncgtaaaacgcctcctccgactcgTCTCTAGG
CTAGAGCGCAGCTCTGGGGGATTACTACTCACTACATCATTAGATCAGATacgactcann
nnnctacgCTATCAGGTCTCGAG
>seq 3
ATCAGCGCTCTATATGGCTCTGATTATAGTTTGCATTCATATGCTGATCTTctcagnntc
cttgacgctcgctATCTGTAGATCTGTACTtcagacagctcTCAGCAGNNNCTCAGCAGC
CTACGACAGTcatgcagactagcagt
\end{verbatim}

Output:
\begin{verbatim}
------------- EXCEPTION -------------
MSG: Each line of the fasta entry must be the same length except the last. Line above #5 'CTAGAGCGCAGCTCTGGGGG..' is 61 != 86 chars...
\end{verbatim}

EMBOSS Seqret was designed as a very flexible tool to convert from one properly formatted file to another properly but distinctly formatted file. It also was designed to accept poorly formatted data (e.g. a FASTA missing the final new line that is improperly wrapped) and export a reformatted file (e.g. wrapped after 60 bases with a final newline). After submitting an inconsistently wrapped FASTA record that is missing a final new line character, seqret (http://www.ebi.ac.uk/Tools/sfc/emboss\_seqret/) produced a properly formatted FASTA record.

Code:

\begin{verbatim}
seqret -stdout -sequence test.fa -outseq test_reformat.fa
\end{verbatim}

Input:

\begin{verbatim}
>my header
AAAAAAAAAAAATTTTTTCCCCGGCGCGCGCGCTATAGCGCTATANNNNNNNNNNNNNNN
ATATATATATAT
ATTATTATATATATATTCTCTCTGGGCTCGCGTCTCGCTATTTATATATATATATATATTGCGCTCTCGTCTCCT\end{verbatim}

Output:

\begin{verbatim}
>my header
AAAAAAAAAAAATTTTTTCCCCGGCGCGCGCGCTATAGCGCTATANNNNNNNNNNNNNNN
ATATATATATATATTATTATATATATATTCTCTCTGGGCTCGCGTCTCGCTATTTATATA
TATATATATATTGCGCTCTCGTCTCCT
\end{verbatim}

However, Seqret did not log the detected errors in the format. Another feature of Seqret is that an output file is created even if the output is identical to the input. Storing two identical files is an inefficient use of disk space. Seqtk is another example of a tool that can automate FASTA reformatting but does not first check original format or report format issues. 

Restarting analysis manually after wrapping a FASTA file may only take minutes but the issue is how long it takes the analyst to become available. Likewise, storage is trivial unless the FASTA files in question store whole genomes in which case the burden can add up for a bioinformatics core. Efficiency and automation a crucial as bioinformatics projects become more numerous and time consuming. Many tools can either detect a format issue or repair a format issue. No existing tool was found that both validates FASTA format and reformats automatically only where required for a user defined list of non-fatal FASTA format issues.

  
  
  
  
  
  
  
  
  
  
  
  
  
  
  
  
  
  
  
  
  
  
  
  
  
  
  
  